\documentclass[10pt,landscape]{article}
\usepackage{multicol}
\usepackage{calc}
\usepackage{ifthen}
\usepackage[landscape]{geometry}
\usepackage{hyperref}

% To make this come out properly in landscape mode, do one of the following
% 1.
%  pdflatex latexsheet.tex
%
% 2.
%  latex latexsheet.tex
%  dvips -P pdf  -t landscape latexsheet.dvi
%  ps2pdf latexsheet.ps


% If you're reading this, be prepared for confusion.  Making this was
% a learning experience for me, and it shows.  Much of the placement
% was hacked in; if you make it better, let me know...


% 2008-04
% Changed page margin code to use the geometry package. Also added code for
% conditional page margins, depending on paper size. Thanks to Uwe Ziegenhagen
% for the suggestions.

% 2006-08
% Made changes based on suggestions from Gene Cooperman. <gene at ccs.neu.edu>


% To Do:
% \listoffigures \listoftables
% \setcounter{secnumdepth}{0}


% This sets page margins to .5 inch if using letter paper, and to 1cm
% if using A4 paper. (This probably isn't strictly necessary.)
% If using another size paper, use default 1cm margins.
\ifthenelse{\lengthtest { \paperwidth = 11in}}
	{ \geometry{top=.5in,left=.5in,right=.5in,bottom=.5in} }
	{\ifthenelse{ \lengthtest{ \paperwidth = 297mm}}
		{\geometry{top=1cm,left=1cm,right=1cm,bottom=1cm} }
		{\geometry{top=1cm,left=1cm,right=1cm,bottom=1cm} }
	}

% Turn off header and footer
\pagestyle{empty}
 

% Redefine section commands to use less space
\makeatletter
\renewcommand{\section}{\@startsection{section}{1}{0mm}%
                                {-1ex plus -.5ex minus -.2ex}%
                                {0.5ex plus .2ex}%x
                                {\normalfont\large\bfseries}}
\renewcommand{\subsection}{\@startsection{subsection}{2}{0mm}%
                                {-1explus -.5ex minus -.2ex}%
                                {0.5ex plus .2ex}%
                                {\normalfont\normalsize\bfseries}}
\renewcommand{\subsubsection}{\@startsection{subsubsection}{3}{0mm}%
                                {-1ex plus -.5ex minus -.2ex}%
                                {1ex plus .2ex}%
                                {\normalfont\small\bfseries}}
\makeatother

% Define BibTeX command
\def\BibTeX{{\rm B\kern-.05em{\sc i\kern-.025em b}\kern-.08em
    T\kern-.1667em\lower.7ex\hbox{E}\kern-.125emX}}

% Don't print section numbers
\setcounter{secnumdepth}{0}


\setlength{\parindent}{0pt}
\setlength{\parskip}{0pt plus 0.5ex}


% -----------------------------------------------------------------------

\begin{document}

\raggedright
\footnotesize
\begin{multicols}{3}


% multicol parameters
% These lengths are set only within the two main columns
%\setlength{\columnseprule}{0.25pt}
\setlength{\premulticols}{1pt}
\setlength{\postmulticols}{1pt}
\setlength{\multicolsep}{1pt}
\setlength{\columnsep}{2pt}

\begin{center}
     \Large{\textbf{HP-16c Cheat Sheet}} \\
\small{\texttt{https://github.com/gvnn3/hp16cheat}}
\end{center}

\section{Key to Symbols}
\begin{tabular}{@{}ll@{}}
\emph{DS} & Drops stack.\\
\emph{LS} & Lifts stack.\\
\emph{SU} & Stack unchanged.\\
\end{tabular}


\section{Clearning}
\begin{tabular}{@{}ll@{}}
\texttt{BSP} & Backspace. \\
\texttt{CLx} & Clear X. \\
\texttt{CLEAR PRGM} & Clear program memory.\\
\texttt{CLEAR REG}  & Clears all registers. \\
\texttt{CLEAR PREFIX}  & Clear any prefix entry.
\end{tabular}

\section{Data Entry}
\begin{tabular}{@{}ll@{}}
\texttt{ENTER} & Copy X into Y.\\
\texttt{CHS} & Change sign.\\
\texttt{EEX} & Enter Exponent (Floating point mode).\\
\end{tabular}

\section{Stack Manipulation}
\begin{tabular}{@{}ll@{}}
\texttt{X<>Y} & Exchange X and Y\\
\texttt{R$\vee$R$\wedge$} & Roll Stack Down/Up\\
\end{tabular}

\section{Display Control}
\begin{tabular}{@{}ll@{}}
\texttt{HEX/DEC/OCT/BIN} & Change number base.\\
\texttt{SHOW HEX/DEC/OCT/BIN} & Show X in another base.\\
\texttt{SET COMPL 1s, 2s, UNSGN} & Set complement mode.\\
\texttt{WSIZE} & Set word size 1..64 (use 0 for 64).\\
\texttt{WINDOW} & 0..7 Display eight digit segm of X\\
\texttt{< >} & Scroll left or right.\\
\texttt{SF/CF N} & Set/Clear flag [0..5].\\
\texttt{STATUS} & Show compl, wordsize and flags.\\
\texttt{FLOAT N} & Choose decimals with 0..9\\
\end{tabular}

\section{Math}
\begin{tabular}{@{}ll@{}}
\texttt{+,-,$\times$,$\div$} & X $\leftarrow$ Y OP X, \emph{DS}\\
\texttt{RMD} & X $\leftarrow$ Y MOD X, \emph{DS}\\
\texttt{$\sqrt{x}$} & Square root of X, \emph{SU}\\
\texttt{$1/\_{x}$} & Reciprocal of X, \emph{SU}\\
\texttt{DBL$\times$,DBL$\div$,DBLR} & Math with doubles\\
\texttt{ABS} & Absolute value of X.
\end{tabular}

\section{Bit Operations}
\begin{tabular}{@{}ll@{}}
\texttt{SL/SR} & Shift left/right \emph{0}\\
\texttt{ASR} & Arithmetic shift right \emph{SGN}\\
\texttt{RL/RR} & Rotate left/right preserving bits.\\
\texttt{RLC/RRC} & Rotate through carry bit.\\
\texttt{RLn/Rn/RLCn/RRCn} & Rotate Y, X bits \emph{DS}.\\
\texttt{LR} & Left justfy X into Y, leaving bit count in X.\\
\texttt{MASKL/MASKR} & Create left of right bit mask based on X.\\.
\texttt{SB/CB} & Set/clear bit in Y based on X.\\
\texttt{\#B} & X $\leftarrow$ sum bits in X \emph{SU}.\\
\texttt{NOT|OR|AND|XOR} & X $\leftarrow$ X OP Y. \emph{DS}.
\end{tabular}

\section{Memory}
\begin{tabular}{@{}ll@{}}
\texttt{STO} & Store value in X into reg 0..F, I, (i).\\
\texttt{RCL} & Recall value from 0..F, I, (i) into X.\emph{LS}\\
\texttt{X<>I} & Exchange X and index register.\\
\texttt{X<>(i)} & Exchange X and register indexed by I.\\
\texttt{LSTx} & Recall previous X into X.\\
\texttt{MEM} & Show memory status.\\
\end{tabular}

\section{Programming}
\begin{tabular}{@{}ll@{}}
\texttt{P/R} & Program or Run mode\\
\texttt{R/S} & Run/Stop\\
\texttt{LBL} & 0..F Set a program label\\
\texttt{RTN} & Return from subroutine or exit program.\\
\texttt{PSE} & Pause and show X.\\
\texttt{GTO} & Goto \texttt{LABEL}.\\
\texttt{GTO .nnn} & Goto line N.\\
\texttt{GSB} & Goto a subroutine.\\
\texttt{SST} & Single step forwards.\\
\texttt{BST} & Single step backwards.\\
\texttt{F?} & If flag \emph{unset}, skip an instruction.\\
\texttt{B?} & If bit \emph{unset}, skip.\\
\texttt{<,$\leq$,$\geq$,>,$\neq$,} & If \emph{false}, skip.\\
\texttt{DSZ/ISZ} & Decrement/increment index, skip if 0.
\end{tabular}

\end{multicols}
\end{document}

%%% Local Variables:
%%% mode: latex
%%% TeX-master: t
%%% End:
